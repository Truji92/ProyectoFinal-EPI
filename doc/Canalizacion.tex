\subsection{Canalización e infraestructura de distribución.}
En este capítulo se definen, dimensionan y ubican las canalizaciones, registros y recintos que constituirán la infraestructura donde se alojarán los cables y equipamiento necesarios para permitir el acceso de los usuarios a los servicios de telecomunicaciones definidos en los capítulos anteriores.
\subsubsection{Consideraciones sobre el esquema general del edificio.}
El esquema general del edificio se refleja en el plano """"meter referencia a plano"""", en él se detalla la infraestuctura necesaria, que comienza, por la parte inferior del edificio en la arqueta de entrada y por la parte superior del edificio en la canalización de enlace superior, y termina en las tomas de usuario. Esta infraestuctura la componen las siguientes partes: arqueta de entrada y canalización externa, canalizaciones de enlace, recintos de instalaciones de telecominicación, registros principales, canalización principal y registros secundarios, canalización secundaria y registros de paso, registros de terminación de red, canalización interior de usuario y registtros de toma, según se describe a continuación.
\subsubsection{Arqueta de entrada y canalización externa.}
Permiten el acceso de los Servicios de Telecomunicaciones de Telefonía Disponible al Público y de Banda Ancha. La arqueta es el punto de convergencia de las redes de alimentación de los operadores de estos servicios, y desde la cual parten los cables de las redes de alimentación de los operadores que discurren por la canalización externa y de enlace hasta el RITU.

\subsubsection*{Arqueta de entrada}
Tendrá unas dimensiones mínimas de 40x40x60 cm (ancho, largo y profundo). Inicialmente se ubicará en la zona indicada en el plano """"incluir referencia a plano"""".
\subsubsection*{Canalización externa}
Estará compuesta por 4 tubos, de 63 mm de diámetro exterior embitidos en un prisma de hormigón y con la siguiente funcionalidad:
\begin{itemize}
	\item 2 Conductos para STDP y TBA.
	\item 2 conductos de reserva.
\end{itemize}
Tanto la construcción de la arqueta de entrada como la de la canalización externa son responsabilidad de la propiedad de la edificación.
Sus características se detallan en el Pliego de condiciones """"añadir referencia a apartado aqui"""".
\subsubsection{Recinto Único.}
Según las características de nuestro proyecto necesitaremos un Recinto de Instalación de Telecomunicación Único (RITU). Consiste en un armario modular donde se ubicará el cuadro de protección eléctrica y los registros principales de cables de pares, cables coaxiales con las regletas y paneles de salida instalados, y en los que se reservará espacio suficiente para las regletas y paneles de entrada a instalar por los operadores que presten sus servicios.
Las dimensiones de este recinto son:

\begin{itemize}
	\item Anchura: 150 cm.
	\item Profundidad: 50 cm.
	\item Altura: 200 cm.
\end{itemize}
Dimensiones accesos: 180x80 cm.

Por la zona inferior del armario acometerán los tubos que forman la canalización de enlace inferior.
Por la zona superior del armario accederán los cables del sistema de captación.
Por la zona inferior del armario acometerán los tubos que forman la canalización principal.

En él quedarán terminados los cables de la red de distribución mediante conectores tipo F y dispondrá de espacio para albergar en su momento los distribuidores y amplificadores que instalen los operadores que presten servicio a través de la red de cables coaxiales.

\subsubsection{Registros principales.}
Los Registros Principales tienen como función albergar el Punto de Interconexión, entre la red exterior y la red interior del inmueble.
Existen tres tipos de Registros Principales: para Red de Cables de Pares Trenzados, para Red de Cables Coaxiales y para Red de Cables de Fibra Óptica.

\subsubsection*{Registro Principal para Red de Cables de Pares Trenzados.}
El Registro principal para Red de Cables de Pares Trenzados es una caja de 500x500x300 (alto x ancho x fondo) mm.
En él se instalará un panel de conexión o panel repartidor de salida y dispondrá de espacio para que los operadores instalen sus paneles de conexión de entrada.

La unión con las regletas o paneles de conexión de entrada se realizará mediante latiguillos de conexión.

Sus características se incluyen en el Pliego de Condiciones.

\subsubsection*{Registro Principal para Red de Cables Coaxiales.}
El Registro Principal para Red de Cables Coaxiales es una caja de 500x500x300 (alto x ancho x fondo) mm.
En él quedarán terminados los cables de la red de distribución mediante conectores tipo F y dispondrá de espacio para albergar en su momento los distribuidores y amplificadores que instalen los operadores que presten servicio a través de la red de bacles coaxiales.

\subsubsection*{Registro Principal para Red de Cables de Fibra Óptica.}
El Registro Principal para Red de Cables de Fibra Óptica es una caja de 500x1000x300 (alto x ancho x fondo) mm.
En él se alojará un panel de conectores de salida constituido por un módulo básico de 48 conectores (24 dobles) y dispondrá de espacio para que los operadores instalen sus paneles de conectores de entrada.
\subsubsection{Canalización principal y registros secundarios.}
Es la que soporta la red de distribución de la ICT del edificio. Une los dos recintos de instalaciones de telecomunicación. Su función es la de alojar las redes de Cables de Pares Trenzados, de Cables Coaxiales, de Cables de Fibra Óptica y red de RTV hasta las diferentes plantas y facilitar la fistribución de los servicios a los usuarios finales.
\subsubsection*{Canalización principal.}
Está compuesta por 6 tubos de 50 mm de diámetro exterior, distribuidos de la siguiente forma:

\begin{tabular}{l l}
	Cables de Pares Trenzados: & 1 x \diameter 50 mm \\
	Cables de Fibra Óptica: & 1 x \diameter 50 mm \\
	Cables Coaxiales para TBA: & 2 x \diameter 50 mm \\
	Cables Coaxiales para RTV: & 1 x \diameter 50 mm \\
	Reserva: & 1 x \diameter 50 mm
\end{tabular}

Sus características se especifican en el Pliego de Condiciones.

\subsubsection*{Registros secundarios}
Son cajas o armarios, que se intercalan en la canalización principal en cada zona y en los cambios de dirección y que sirven para poder segregar en la misma todos los servicios en número suficiente para los usuarios de esa zona. La canalización principal entra por uno de los lados y sale por el contrario.
De ellos salen los tubos que configuran la canalización secundaria.
Sus dimensiones mínimas serán: 45x45x15 cm (anchura, altura, profundidad).
Dentro se colocan los dos derivadores de los dos ramales de RTV, las regletas para la segregación de pares telefónicos y las cajas de segregación de los cables de fibra óptica.

Sus características se especifican en el Pliego de Condiciones.
Existirá uno por cada vivienda.

El total de Registros Secundarios necesarios es de:
13 Registros Secundarios de 45x45x15 cm (anchura, altura, profundidad).
\subsubsection{Canalización secundaria y registros de paso.}
\subsubsection*{Canalización secundaria}
Es la que soporta la red de dispersión. Conecta los registros secundarios con los registros de terminación de red en el interior de las viviendas o locales comerciales.

Está formada por 3 tubos que van directamente desde cada RS al RTR de cada vivienda con la siguiente funcionalidad y diámetro exterior:

\begin{tabular}{l}
1 de \diameter 25 mm. para alojar el cable de pares trenzados y el de fibra óptica.\\
1 de \diameter 25 mm. para alojar el cable coaxial de TBA.\\
1 de \diameter 25 mm. para alojar los dos cables coaxiales de RTV.\\
\end{tabular}


Sus características se especifican en el Pliego de Condiciones.

\subsubsection*{Registros de paso}
Debido a la canalizacion de los cables en el interior de las viviendas a través de las paredes, rodeando a la casa, son necesarios dos Registros de paso en las esquinas de ambas plantas.

Serán necesario instalar cajas de tipo B de tamaño 10x10x4 cm (alto, ancho, fondo).

\subsubsection{Registros de Terminación de Red.}
Conectan la red de dispersión con la red interior de usuario. En estos registros se alojan los puntos de acceso de usuario (PAU) de los distintos servicios, que separan la red comunitaria de la privada de cada usuario.

Estarán constituidos por cajas empotradas en la pared de vivienda o local provistas de tapa y sus dimensiones mínimas serán de 50x60x8 cm (alto, ancho, fondo).

Sus características se especifican en el Pliego de Condiciones.

Los registros de terminación de red dispondrán de tres tomas de corriente o bases de enchufe.

El total de Registros de Terminación de red necesarios es de 11.

\subsubsection{Canalización Interior de Usuario.}
Es la que soporta la red interior de usuario. Está realizada por tubos, empotrados por el interior de la vivienda que unen el RTR con los distintos Registros de Toma.

La topología de las canalizaciones será en estrella.

El diámetro de los tubos será:\\
De \diameter 2 cm. para Cables de Pares Trenzados.\\
De \diameter 2 cm. para Cable Coaxial de TBA. \\
De \diameter 2 cm. para Cablel Coaxial de RTV. \\

Sus características se especifican en el Pliego de Condiciones.

Ver plano """"METER REFERENCIA A PLANO AQUI""""		

\subsubsection{Registros de toma.}
Son cajas empotradas en la pared donde se alojan las bases de acceso terminal (BAT) o tomas de usuario. Las dimensiones mínimas son 6,4x6,4x4,2 cm (alto, ancho, fondo).

En las viviendas se instalarán en el salón-comedor y en el dormitorio principal dos registros de toma para cables de pares trenzados, un registro para toma de cables coaxiales para servicios de TBA y un registro para toma de cables coaxiales para servicios de RTV.

En los otros dos dormitorios y en la cocina se instalará un registro para toma de cable de pares trenzados y un registro para toma de cable coaxial para servicios de RTV.

En las proximidades del RTR se situará un registro para una toma configurable.

En los locales no se instalarán registros de toma.

La ubicación de los registros de toma en cada estancia se indica en el plano """"REFERENCIA A PLANO AQUI""""	

El total de registros de toma a instalar será de 80 (de los cuales 10 serán configurables).

Las características de los Registros de Toma se especifican en el Pliego de Condiciones.
