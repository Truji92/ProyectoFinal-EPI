\subsection{Canalización e infraestructura de distribución.}
En este capítulo se definen, dimensionan y ubican las canalizaciones, registros y recintos que constituirán la infraestructura donde se alojarán los cables y equipamiento necesarios para permitir el acceso de los usuarios a los servicios de telecomunicaciones definidos en los capítulos anteriores.
\subsubsection{Consideraciones sobre el esquema general del edificio.}
El esquema general del edificio se refleja en el plano """"meter referencia a plano"""", en él se detalla la infraestuctura necesaria, que comienza, por la parte inferior del edificio en la arqueta de entrada y por la parte superior del edificio en la canalización de enlace superior, y termina en las tomas de usuario. Esta infraestuctura la componen las siguientes partes: arqueta de entrada y canalización externa, canalizaciones de enlace, recintos de instalaciones de telecominicación, registros principales, canalización principal y registros secundarios, canalización secundaria y registros de paso, registros de terminación de red, canalización interior de usuario y registtros de toma, según se describe a continuación.
\subsubsection{Arqueta de entrada y canalización externa.}
Permiten el acceso de los Servicios de Telecomunicaciones de Telefonía Disponible al Público y de Banda Ancha. La arqueta es el punto de convergencia de las redes de alimentación de los operadores de estos servicios, y desde la cual parten los cables de las redes de alimentación de los operadores que discurren por la canalización externa y de enlace hasta el RITU.
\begin{itemize}
	\item Arqueta de entrada
	Tendrá unas dimensiones mínimas de 40x40x60 cm (ancho, largo y profundo). Inicialmente se ubicará en la zona indicada en el plano """"incluir referencia a plano"""".
	
	\item Canalización externa
	Estará compuesta por 4 tubos, de 63 cm de diámetro exterior embutidos en un prisma de hormigón y con la siguiente funcionalidad:
	\begin{itemize}
	\item 
\end{itemize}
\end{itemize}
\subsubsection{Registros principales.}
\subsubsection{Canalización principal y registros secundarios.}
\subsubsection{Canalización secundaria y registros de paso.}
\subsubsection{Registros de terminación de red.}
\subsubsection{Canalización interior de usuario.}
\subsubsection{Registros de toma.}
\subsubsection{Cuadro Resumen de materiales necesarios.}
\paragraph{Arquetas}
\paragraph{Tubos de diverso diámetro y canales.}
\paragraph{Registros de los diversos tipos.}
\paragraph{Material de equipamiento de los recintos.}