\subsection{Canalización e infraestructura de distribución.}
En este capítulo se definen, dimensionan y ubican las canalizaciones, registros y recintos que constituirán la infraestructura donde se alojarán los cables y equipamiento necesarios para permitir el acceso de los usuarios a los servicios de telecomunicaciones definidos en los capítulos anteriores.
\subsubsection{Consideraciones sobre el esquema general del edificio.}
El esquema general del edificio se refleja en el plano """"meter referencia a plano"""", en él se detalla la infraestuctura necesaria, que comienza, por la parte inferior del edificio en la arqueta de entrada y por la parte superior del edificio en la canalización de enlace superior, y termina en las tomas de usuario. Esta infraestuctura la componen las siguientes partes: arqueta de entrada y canalización externa, canalizaciones de enlace, recintos de instalaciones de telecominicación, registros principales, canalización principal y registros secundarios, canalización secundaria y registros de paso, registros de terminación de red, canalización interior de usuario y registtros de toma, según se describe a continuación.
\subsubsection{Arqueta de entrada y canalización externa.}
Permiten el acceso de los Servicios de Telecomunicaciones de Telefonía Disponible al Público y de Banda Ancha. La arqueta es el punto de convergencia de las redes de alimentación de los operadores de estos servicios, y desde la cual parten los cables de las redes de alimentación de los operadores que discurren por la canalización externa y de enlace hasta el RITU.

\subsubsection*{Arqueta de entrada}
Tendrá unas dimensiones mínimas de 40x40x60 cm (ancho, largo y profundo). Inicialmente se ubicará en la zona indicada en el plano """"incluir referencia a plano"""".
\subsubsection*{Canalización externa}
Estará compuesta por 4 tubos, de 63 mm de diámetro exterior embitidos en un prisma de hormigón y con la siguiente funcionalidad:
\begin{itemize}
	\item 2 Conductos para STDP y TBA.
	\item 2 conductos de reserva.
\end{itemize}
Tanto la construcción de la arqueta de entrada como la de la canalización externa son responsabilidad de la propiedad de la edificación.
Sus características se detallan en el Pliego de condiciones """"añadir referencia a apartado aqui"""".
\subsubsection{Recinto Único.}
Según las características de nuestro proyecto necesitaremos un Recinto de Instalación de Telecomunicación Único (RITU). Consiste en un armario modular donde se ubicará el cuadro de protección eléctrica y los registros principales de cables de pares, cables coaxiales con las regletas y paneles de salida instalados, y en los que se reservará espacio suficiente para las regletas y paneles de entrada a instalar por los operadores que presten sus servicios.
Las dimensiones de este recinto son:
\begin{itemize}
	\item Anchura: 150 cm.
	\item Profundidad: 50 cm.
	\item Altura: 200 cm.
\end{itemize}
Dimensiones accesos: 180x80 cm.

Por la zona inferior del armario acometerán los tubos que forman la canalización de enlace inferior.
Por la zona superior del armario accederán los cables del sistema de captación.
Por la zona inferior del armario acometerán los tubos que forman la canalización principal.

En él quedarán terminados los cables de la red de distribución mediante conectores tipo F y dispondrá de espacio para albergar en su momento los distribuidores y amplificadores que instalen los operadores que presten servicio a través de la red de cables coaxiales.

""""Meter tablita aqui""""

\subsubsection{Registros principales.}
Los Registros Principales tienen como función albergar el Punto de Interconexión, entre la red exterior y la red interior del inmueble.
Existen tres tipos de Registros Principales: para Red de Cables de Pares/Pares Trenzados, para Red de Cables Coaxiales y para Red de Cables de Fibra Óptica.

\subsubsection*{Registro Principal para Red de Cables de Pares Trenzados.}
El Registro principal para Red de Cables de Pares Trenzados es una caja de 500x500x300 (alto x ancho x fondo) mm.
En él se instalará un panel de conexión o panel repartidor de salida y dispondrá de espacio para que los operadores instalen sus paneles de conexión de entrada.

La unión con las regletas o paneles de conexión de entrada se realizará mediante latiguillos de conexión.

Sus características se incluyen en el Pliego de Condiciones.

\subsubsection*{Registro Principal para Red de Cables Coaxiales.}
El Registro Principal para Red de Cables Coaxiales es una caja de 500x500x300 (alto x ancho x fondo) mm.
En él quedarán terminados los cables de la red de distribución mediante conectores tipo F y dispondrá de espacio para albergar en su momento los distribuidores y amplificadores que instalen los operadores que presten servicio a través de la red de bacles coaxiales.

\subsubsection*{Registro Principal para Red de Cables de Fibra Óptica.}
El Registro Principal para Red de Cables de Fibra Óptica es una caja de 500x1000x300 (alto x ancho x fondo) mm.
En él se alojará un panel de conectores de salida constituido por un módulo básico de 48 conectores (24 dobles) y dispondrá de espacio para que los operadores instalen sus paneles de conectores de entrada.
\subsubsection{Canalización principal y registros secundarios.}
\subsubsection{Canalización secundaria y registros de paso.}
\subsubsection{Registros de terminación de red.}
\subsubsection{Canalización interior de usuario.}
\subsubsection{Registros de toma.}
\subsubsection{Cuadro Resumen de materiales necesarios.}
\paragraph{Arquetas}
\paragraph{Tubos de diverso diámetro y canales.}
\paragraph{Registros de los diversos tipos.}
\paragraph{Material de equipamiento de los recintos.}