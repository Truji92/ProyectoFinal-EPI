\subsection{Distribución de los servicios de telecomunicaciones de telefonía disponible al público (STDP) y de banda ancha (TBA).}
Será responsabilidad de la propiedad de la edificación, el diseño e instalación de las redes de distribución, dispersión e interior de usuario de estos servicios.
\subsubsection{Redes de cables de pares o pares trenzados.}
\paragraph{Características de los cables.}
Los cables de pares trenzados se utilizan en la red de distribución y dispersión y en la red interior de usuario.

Para las redes de distribución y dispersión, los cables de pares trenzados utilizados serán, como mínimo, de 4 pares de hilos conductores de cobre con aislamiento individual sin apantallar clase E (categoría 6), deberán cumplir las especificaciones de la norma UNE-EN 50288-6-1 (Cables metálicos con elementos múltiples utilizados para la transmisión y el control de señales analógicas y digitales. Parte 6-1: Especificación intermedia para cables sin apantallar aplicables hasta 250 MHz. Cables para instalaciones horinzontales y verticales en edificios).

Para la red interior de usuario, los cables utilizados serán como mínimo de cuatro pares de hilos conductores de cobre con aislamiento individual clase E (categoría 6) y cubierta de material no propagador de la llama, libre de halógenos y baja emisión de humos, y deberán ser conformes a las especificaciones de la norma UNE-EN 50288-6-1 (Cables metálicos con elementos múltiples utilizados para la transmisión y el control de señales analógicas y digitales. Parte 6-1: Especificación intermedia para cables sin apantallar aplicables hasta 250 MHz. Cables para instalaciones horizonttales y verticales en edificios) y UNE-EN 50288-6-2 (Cables metálicos con elementos múltiples utilizados para la transmisión y el control de señales analógicas y digitales. Parte 6-2: Especificación intermedia para cables sin apantallar aplicables hasta 250 MHz. Cables para instalaciones en el área de trabajo y cables para conexionado). 

Las redes de distribución y dispersión deberán cumplir los requisitos especificados en las normas UNE-EN 50174-1:2001 (Tecnología de la información. Instalación del cableado. Parte 1: Especificación y aseguramiento de la calidad), UNE-EN 50174-2 (Tecnología de la información. Instalación del cableado. Parte 2: Métodos y planificación de la instalación en el interior de los edificios) y UNE-EN 50174-3 (Tecnología de la información. Instalación del cableado. Parte 3: Métodos y planificación de la instalación en el exterior de los edificios) y serán certificadas con arreglo a la norma UNE-EN 50346 (Tecnologías de la información. Instalación de cableado. Ensayo de cableados instalados).

Los cables de pares trenzados que se utilizarán en este proyecto deberán tener una atenuación máxima de 34 dB/100 metros a 300 MHz y serán de categoría 6 de clase E o superior.
\paragraph{Características de los elementos activos.}
No se instalarán elementos activos en la red de pares trenzados.
\paragraph{Características de los elementos pasivos.}
Los elementos de conexión (regletas y conectores) de pares metálicos cumplirán los siguientes requisitos eléctricos:
\begin{itemize}
	\item La resisterncia de aislamiento entre contactos, en condiciones normales ($23^{\circ}C$, 50\% H.R.), deberá ser superior a $10^{6} M\Omega.$ 
	\item La resistencia de contacto con el punto de conexión de los cables/hilos deberá ser inferior a 10 $m\Omega.$ 
	\item La rigidez dieléctrica deberá ser tal que soporte una tensión entre contactos de 1.000 $V_{efca}\pm10\%$ y $1.500 V_{cc}\pm10\%$.
\end{itemize}

\subparagraph{Panel de conexión para cables de pares trenzados.}
El panel de conexión para cables de pares trenzados, en el punto de interconexión, alojará tantos puertos como cables que constituyen la red de distribución. Cada uno de estos puertos tendrá un lado preparado para conectar los conductores de cable de la red de distribución, y el otro lado estrá formado por un conector hembra miniatura de 8 vías (RJ45) de tal forma que en el mismo se permita el conexionado de los cables dde acometida de la red de alimentación o de los latiguillos de interconexión. Los conectores cumplirán la norma UNE-EN 50173-1 (Tecnología de la información. Sistemas de cableado genérico. Parte 1: Requisitos generales y áreas de oficina).

El panel que aloja los puertos indicados será de material plástico o metálico, permitiendo la fácil inserción-extracción en los conectores y la salida de los cables de la red de distribución.

\subparagraph{Punto de Acceso al Usuario (PAU)}
El conector de la roseta de terminación de los cables de pares trenzados será un conector hembra miniatura de 8 vías (RJ45) con todos los contactos conexionados. Este conector cumplirá las normas UNE-EN 50173-1 (Tecnología de la información. Sistemas de cableado genérico. Parte 1: Requisitos generales y áreas de oficina).

\subparagraph{Conectores para Cables de Pares Trenzados}
Las diferentes ramas de la red interior de usuario partirán del interior del PAU equipados con conectores macho miniatura de ocho vías (RJ45) dispuestas para cumplir la norma UNE-EN 50173-1 (Tecnología de la información. Sistemas de cableado genérico. Parte 1: Requisitos generales y áreas de oficina).

Las bases de acceso de los terminales estarán dotadas de uno o varios conectores hembra miniatura de ocho vías (RJ45). dispuestas para cumplir la citada norma.

\subsubsection{Redes de cables coaxiales.}
\paragraph{Características de los cables.}
Con carácter general, los cables coaxiales a utilizar en las redes de distribución y dispersión serán de los tipos RG-6, RG-11 y RG-59.

Los cables coaxiales cumplirán con las especificaciones de las Normas UNE-EN 50117-2-1 (Cables  coaxiales. Parte 2-1: Especificación intermedia para cables utilizados en redes de distribución por cable. Cables de interior para la conexión de sistemas funcionando entre 5 MHz y 1.000 MHz) y de la norma UNE-EN 50117-2-2 (Cables coaxiales. Parte 2-2: Especificación intermedia para cables utilizados en redes de distribución cableadas. Cables de acometida exterior para sistemas operando entre 5 MHz y 1.000 MHz) y cumpliendo:
\begin{itemize}
	\item Impedancia característica media 75 Ohmios.
	\item Conductor central de acero recubierto de cobre de acuerdo a la Norma UNE-EN 50117-1.
	\item Dieléctrico de polietileno celular fésico, expandido mediante inyección de gas de acuerdo a la norma UNE-EN 50290-2-23, estando adherido al conductor central.
	\item Pantalla formada por una cinta laminada de aluminio-poliester-aluminio solapada y pegada sobre el dieléctrico.
	\item Malla formada por una trenza de alambres de aluminio, cuyo porcentaje de recubrimiento será superior al 75\%.
	\item Cubierta externa de PVC, resistente a rayos ultravioletas para el exterior, y no propagador de la llama debiendo cumplir la normativa UNE-EN 50265-2 de resistencia de propagación de la llama.
	\item Cuando sea necesario, el able deberá estar dotado con un compuesto anti-humedad contra la corrosión, asegurando su estanqueidad logitudinal.
\end{itemize}

Los diámetros exteriores y atenuación máxima de los cables cumplirán:

\begin{tabular}{| p{4cm} | c | c | c |}
	\hline
	& RG-11 & RG-6 & RG-59 \\
	\hline
	Diámetro exterior (mm) & $10.3\pm0.2$ & $7.1\pm0.2$ & $6.2\pm0.2$\\
	\hline
	Atenuaciones & dB/100m & dB/100m & dB/100m\\
	\hline
	5 MHz & 1.3 & 1.9 & 2.8 \\
	\hline
	862 MHz & 13.5 & 20 & 24.5\\
	\hline
	Atenuación de apantallamiento & \multicolumn{3}{| p{8cm} |}{Clase A según Apartado 5.1.2.7 de las Normas UNE-EN 50117-2-1 y UNE-EN 50117-2-2}\\
	\hline
\end{tabular}
\paragraph{Características de los elementos pasivos.}
Todos los elementos pasivos de exterior permitirán el paso y corte de corriente incluso cuando la tapa esté abierta, la cual estará equipada con una junta de neopreno o de poliuterano y de una malla metálica, que aseguren tanto su estanqueidad como su apantallamiento electromagnético. Los elementos pasivos de interior no permitirán el paso de corriente.

Todos los elementos pasivos utilizados en la red de cables coaxiales tendrán una impedancia nominal de $75\Omega$, con unas pérdidas de retorno superiores a 15 dB en el margen de frecuencias de funcionamiento de los mismos que, al menos, estará comprendido entre 5 MHz y 1.000 MHz, y estarán diseñados de forma que permitan la transimisión de señales en ambos sentidos simultáneamente.

La respuesta amplitud-frecuencia de los derivadores cumplirá lo dispuesto en la norma UNE-EN 50083-4 (Redes de distribución por cable para señales de televisión, sonido y servicios interactivos. Parte 4: Equipos pasivos de banda ancha utilizados en ls redes de distribución coaxial), tendrán una directividad superior a 10 dB, un aislamiento derivación-salida superior a 20 dB y su aislamiento electromagnético cumplirá lo dispuesto en la norma UNE-EN 50083-2 (Redes de distribución por cable para señales de televisión, señales de sonido y servicios interactivos. Parte 2: Compatibilidad electromagnética de los equipos).

Todos los puertos de los elementos pasivos estarán dotados con conectores tipo F y la base de los mismos dispondrá de un herraje para la fijación del dispositivo en la pared. Su diseño será tal que asegure el apantallamiento electromagnético y, en el caso de los elementos pasivos de exterior, la estanquedad del dispositivo.

\subparagraph{Cargas tipo F inviolables.}
Estarán constituidas por un cilindro formado por una pieza única de material de alta resistencia a la corrosión. El puerto de entrada F tendrá una espiga para la instalación en el puerto F hembra del derivador. La rosca de conexión será de 3/8-32.

\subparagraph{Cargas de terminación}
La carga de terminación coaxial a instalar en todos los puertos de los derivadores o distribuidores (incluidos los de terminación de línea) que no lleven conectado un cable de acometida será una carga de 75 ohmios de tipo F.

\subparagraph{Conectores}
Con carácter general en la red de cables coaxiales se turilizarán conectores de tipo F universal de compresión.

\subparagraph{Distribuidor}
Estará constituido por un distribuidor simétrico de dos salidas equipadas con conectores del tipo F hembra.

\subparagraph{Bases de acceso de Terminal}
Cumplirán las siguientes características:

\begin{itemize} 
	\item Características físicas: según normas UNE 20523-7 (Instalaciones de antenas colectivas. Caja de toma), UNE 20523-9 (Instalaciones de antenas colectivas. Prolongador) y UNE-EN 50083-2 (Redes de distribución por cable para señales de televisión, señales de sonido y servicios interactivos. Parte 2: Compatibilidad electromagnética de los equipos).
	\item Impedancia: $75\Omega$
	\item Banda de frecuencia: 86-862 MHz.
	\item Banda de retorno 5-65 MHz.
	\item Pérdidas de retorno TV (40-862 MHz): $>=$14dB-1,5dB/Octava y en todo caso $>=$ 10 dB.
	\item Pérdidas de retorno radiodifusión sonora FM: $>=$ 10dB.
\end{itemize}
\subsubsection{Redes de cables de fibra óptica.}
\paragraph{Características de los cables.}
El cable de acometida óptica será individual de 2 fibras ópticas con el siguiente código de colores: Fibra 1: verde, Fibra 2: roja.

Las fibras ópticas que se utilizarán serán monomodo del tipo G.657 categoría A2 o B3, con baja sensibilidad a curvaturas y están definidas en la recomendación UIT-T G.657 "Características de las fibras y cables ópticos monomodo insensibles aa la pérdida por flexión para la rd de acceso2. Las fibras ópticas deberán ser compatibles con las del tipo G.652.D, definidas en la recomendación UIT-T G.652 "Características de las fibras ópticas y los cables monomodo".

El cable deberá ser completamente dieléctrico, no poseerá nungún elemento metálico y el material de la cubierta de los cables debe ser termoplástico, libre de halógenos, retardante a la llama y de baja emisión de humos.

En lo relativo a los elementos de reguerzo, deberán ser suficientes para garantizar que para una tracción de 450 N, no se producen alargamientos permanentes de las fibras ópticas ni aumentos de la atenuación. Su diámetro estará en torno a 4 milímetros y su radio de curvatura mínimo deberá ser 5 veces el diámetro (2cm).

Se comprobará la continuidad de las fibras ópticas de las redes de distribución y dispersión y su correspondencia con las etiquetas de las regletas o las ramas, mediante un generador de señales ópticas en las longitudes de onda(1310nm, 1490nm y 1550nm) en un extremo y un detector o medidor adecuado en el otro extremo.

Las medidas se realizarán desde las regletas de salida de fibra óptica, situadas en el registro principal óptico del RITU, hasta los conectores ópticos de la roseta de los PAU situada en el RTR de cada vivienda o local.

La atenuación óptica de la red de distribución y dispersión de fibra óptica no deberá ser superior a 2 dB en nungún caso, recomendándose que no supere 1,55 dB.
\paragraph{Características de los elementos pasivos.}
\subparagraph{Caja de interconexión de cables de fibra óptica.}
La caja de interconexión de cables de fibra óptica estará situada en el RITU, y constituirá la realización física del punto de interconexión y desarrollará las funciones de registro principal óptico. La caja se realizará en dos tipos de módulos:
\begin{itemize} 
	\item Módulo de salida para terminar la red de fibra óptica de las casas (uno o varios).
	\item Módulo de entrada para terminar las redes de alimentación de los operadores (uno o varios).
\end{itemize}

El módulo básico para terminar la red de fibra óptica de las casas permitirá la terminación de 48 conectores en regletas donde se instalarán las fibras de la red de distribución terminadas en un conector SC/APC con su correspondiente adpatador. Se instalarán tantos módulos como sean necesarios para atender a la totalidad de la red de distribución.

Las cajas que los alojan estarán dotadas con los elementos pasacables necesarios para la introducción de los cables en las mismas.

Los módulos de terminación de red óptica deberán haber superado als pruebas de frío, calor seco, ciclos de temperatura, humedad y niebla salina, de acuerdo a la parte correspondiente de la familia de normas UNE-EN 60068-2 (Ensayos ambientales. Parte 2: ensayos).

Si las cajas son de material plástico, deberán cumplir la prueba de autoextinguibilidad y haber superado las pruebas de resistencia frente a líquidos y polvo de acuerdo a las normas UNE 20324 (Grados de protección proporcionados por las envolventes, Código IP), donde el grado de protrcción exigido será IP 55. También, deberán haber superado la prueba de impacto de acuerdo a la norma UNE-EN 50102 (Grados de protección proporcionados por las envolventes de materiales eléctricos contra los impactos mecánicos externos, Código IK), donde el grado de protección exigido será IK 08.

Finalmente, las cajas deberán haber superado las pruebas de carga estática, flexión, carga axial en cables, vibración, torsión y durabilidad, de acuerdo con la parte correspondiente de la familia de normas UNE-EN 61300-2 (Dispositivos de interconexión de fibra óptica y componentes pasibos - Ensayos básicos y procedimientos de medida. Parte 2: ensayos).

\subparagraph{Caja de segregación de cables de fibra óptica.}
Las fibras de la red de distribución/dispersión estarán en paso en el punto de distribución. El punto de distribución estará formado por una o varias cajas de segregación en las que se dejarán almacenados, únicamente, los bucles de las fibras ópticas de reserva, con la longitud suficiente para poder llegar hasta el PAU más alejado. Los extremos de las fibras ópticas de la red de dispersión se identificarán mediante etiquetas que indicarán los puntos de acceso al usuario a los que dan servicio.

La caja de segregación de fibras ópticas estará situada en los registros segundarios, y constituirá la realización flísica del punto de distribución óptico. Las cajas de segregación serán de interior, para 8 fibras ópticas.

Las cajas deberán haber superado las mismas pruebas de frío, calor seco, ciclos de temperatura, humedad y niebla salina, de autoextinguibilidad, de resistencia frente a líquidos y polvo (grado de protección exigido será IP 52), grado de protección IK 08, y de pruebas de carga estática, impacto, flexión, carga exial en cables, vibración, torsión y durabilidad, de acuerdo con la parte correspondiente de la familia de normas UNE-EN 61300-2 (Dispositivos de interconexión de fibra óptica y componentes pasivos - Ensayos básicos y procedimientos de medida. Parte 2: ensayos).

Todos los elementos de la caja de segregación estarán diseñados de forma que se garantice un radio de curvatura mínimo de 15 mm en el recorrido de la fibra óptica dentro de la caja.

\subparagraph{Roseta de fibra óptica.}
La roseta para cables de fibra óptica estará situada en el RTR y estará formada por una caja que, a su vez, contendrá o alojará los conectores ópticos SC/APC de terminación en la red de dispersión de fibra óptica.

Las rosetas deberán haber superado las misma pruebas de frío, calor seco, ciclos de temperatura, humedad y niebla salina, de autoextinguibilidad, de resistencia frente a líquidos y polvo (grado de protección exigido será IP 52), y de pruebas de carga estática, impacto, flexión, carga exial en cables, vibración, torsión y durabilidad, de acuerdo con la parte correspondiente de la familia de normas UNE-EN 61300-2 (Dispositivos de interconexión de fibra óptica y componentes pasivos - Ensayos básicos y procedimientos de medida. Parte 2: ensayos).

Cuando la roseta óptica esté equipada con un rabillo para ser empalmado a las acometidas de fibra óptica de la red de distribución, el rabillo con conector que se vaya a posicionar en el PAU será de fibra óptica optimizada frente a curvaturas, del tipo G.657, categoría A2 o B3, y el empalme y los bucles de las fibras ópticas irán alojados en una caja.

Todos los elementos de la caja estarán diseñados de forma que se garantice un radio mínimo de curvatura de 20 mm en el recorrido de la fibra óptica dentro de la caja.

La caja de la rosete óptica estará diseñada para alojar dos conectores ópticos, como mínimo, con sus correspondientes adaptadores.

\subparagraph{Conectores para cables de fibra óptica.}
Los conectores para cables de fibra óptica serán de tipo SC/APC con su correspondiente adaptador, para ser instalados en los paneles de conexión preinstalados en el punto de interconexión del registro principal óptico y en la roseta óptica del PAU, donde irán equipados con los correspondientes adaptadores. Las características de los conectores ópticos responderán al proyecto de norma PNE-prEN 50377-4-2.

Las características ópticas de los conectores ópticos, en relación con la familia de normas UNE-EN 61300-2 (Dispositivos de interconexión de fibra óptica y componentes pasivos - Ensayos básicos y procedimientos de medida. Parte 2: ensayos), serán las siguientes:

\begin{tabular}{| p{4cm}| c | p{3cm} |}
	\hline
	Ensayo & Método de ensayo & Requisitos \\
	\hline
	Atenuación (At) frente a conector de referencia & UNE-EN 61300-3-4 método B & media<=0,30dB máxima<=0,50dB \\
	\hline
	Atenuación (At) de una conexión aleatoria & UNE-EN 61300-3-34 & media<=0,30dB máxima<=0,60dB\\
	\hline
	Pérdida de Retorno (PR) & UNE-EN 61300-3-6 método 1 & APC>=60dB\\
	\hline
\end{tabular}
\paragraph{Características de los empalmes de fibra en la instalación.}
En esta instalación no se realizarán empalmes en las redes de fibra óptica, al realizarse las redes de distribución y dispersión mediante cables de dos fibras desde el RITU hasta cada RTR.
