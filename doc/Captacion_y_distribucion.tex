\subsection{Captación y distribución de radiodifusión sonora y televisión terrestres.}

\subsubsection{Consideraciones sobre el Diseño.}


Las antenas para la recepción de las señales de radiodifusión terrestre y recepción satélite se instalarán sobre el tejado del RITU (ver planos \ref{} AÑADIR REFERENCIA).

Se utilizarán cinco antenas, dos para satélite, dos para radio (Satelite, VHF y Terrestre, FM B-II) y una para televisión.

Los canales serán amplificados en cabecera mediante amplificadores monocanales con objeto de evitar la intermodulación entre ellos. Su figura de ruido, ganancia y nivel máximo de salida se han seleccionado para garantizar en las tomas de usuarios los niveles de calidad exigidos por el Real Decreto 346/2011. Con objeto de reducir el volumen, peso y coste de la cabecera terrestre, los
cuatro canales adyacentes del servicio DAB y los cuatro digitales más elevados (canales 66 a 69), también adyacentes, serán amplificados mediante sendos amplificadores de grupo.

Las redes de distribución y dispersión se han diseñado para obtener el mayor equilibrio posible entre las distintas tomas de usuario con los elementos de red establecidos en el correspondiente apartado del pliego de condiciones.

Siguiendo lo establecido en el Anexo I del Real Decreto 346/2011 las redes de TV se han diseñado con una estructura en estrella colocando a la salida del PAU un distribuidor de tantas vías como estancias (sin incluir baños y trasteros) existen en la vivienda.

El promotor ha definido la existencia de un local comercial pero sin facilitar la distribución interior. Puesto que se carece de esa información se equipará un PAU pero no se instalará distribuidor ni tomas.

\subsubsection{Número de tomas.}

\begin{table}[H]
\caption{Número de tomas de RTV}
\label{tomasRTV}
\begin{center}
\begin{tabular}{|c|c|c|}
\hline
	 & Número de estancias/vivienda & Número de tomas\\
\hline
	Planta baja & 2 & 2\\
\hline
	Primera Planta & 5 & 5\\
\hline
    Local comercial & 0 & 0\\
\hline
\end{tabular}
\end{center}
\end{table}


\begin{table}[H]
\caption{Número total de tomas de RTV}
\label{tomasRTVtotal}
\begin{center}
\begin{tabular}{|c|c|}
\hline
    Total tomas en Viviendas & 70\\
\hline 
    Total tomas en locales comerciales & 0\\
\hline
    Total de tomas & 70	\\
\hline
\end{tabular}
\end{center}
\end{table}

El número total de tomas es de 70 en viviendas. No existen estancias comunes en la edificación.

Según lo dispuesto en el apartado 3.5.2 del Anexo I del Reglamento de ICT, en cada local se colocará un PAU capaz de alimentar un número de tomas fijado en función de la superficie o división interior del los locales. En nuestro caso al no estar definida la división interior, no se colocarán tomas. El diseño y dimensionamiento de la red interior de usuario, así como su instalación futura, será responsabilidad de la propiedad del local, cuando se ejecute el proyecto de su distribución en estancias.

\paragraph{Número de repartidores, derivadores, según su ubicación en la red, PAU y sus características, así como las de los cables utilizados.}

Las redes de distribución y dispersión están formadas por una estructura árbol-rama.
La red de distribución comienza a la salida del mezclador de señales terrestres y satélites y finaliza en el derivador del local.
Las redes interiores tendrán estructura de estrella.


\subparagraph*{Derivadores, PAUs y Repartidores interiores de viviendas y locales}

Cada dos viviendas se colocará un derivador de dos salidas.

En cada vivienda se colocará, a la salida del PAU un distribuidor de 7 salidas. 

A ellas se conectarán los cables de la red interior de usuario correspondientes a cada estancia.

En el local no se instalará distribuidor, instalándose únicamente el PAU.

\subparagraph*{Cables}

Se utilizará un cable de 7 mm de diámetro exterior que deberá cumplir la norma UNE-EN 50117-2-4.

Sus características se especifican en el Pliego de Condiciones.

\subparagraph*{Tomas}

En cada vivienda el número de tomas instaladas es de 7.

En el local comercial, no se instalarán tomas.

No hay estancias comunes en la edificación.

Las caráteristicas técnicas de todos estos elementos se incluyen en el Pliego de Condiciones.