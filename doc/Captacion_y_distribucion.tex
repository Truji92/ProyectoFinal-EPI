\subsection{Captación y distribución de radiodifusión sonora y televisión terrestres.}

\subsubsection{Consideraciones sobre el Diseño.}


Las antenas para la recepción de las señales de radiodifusión terrestre y recepción satélite se instalarán sobre el tejado del RITU (ver planos \ref{} AÑADIR REFERENCIA).

Se utilizarán cuatro antenas, una para satélite, dos para radio (Satelite, VHF y Terrestre, FM B-II) y una para televisión.

Los canales serán amplificados en cabecera mediante amplificadores monocanales con objeto de evitar la intermodulación entre ellos. Su figura de ruido, ganancia y nivel máximo de salida se han seleccionado para garantizar en las tomas de usuarios los niveles de calidad exigidos por el Real Decreto 346/2011. Con objeto de reducir el volumen, peso y coste de la cabecera terrestre, los
cuatro canales adyacentes del servicio DAB y los cuatro digitales más elevados (canales 66 a 69), también adyacentes, serán amplificados mediante sendos amplificadores de grupo.

Las redes de distribución y dispersión se han diseñado para obtener el mayor equilibrio posible entre las distintas tomas de usuario con los elementos de red establecidos en el correspondiente apartado del pliego de condiciones.

Siguiendo lo establecido en el Anexo I del Real Decreto 346/2011 las redes de TV se han diseñado con una estructura en estrella colocando a la salida del PAU un distribuidor de tantas vías como estancias (sin incluir baños y trasteros) existen en la vivienda.

El promotor ha definido la existencia de un local comercial pero sin facilitar la distribución interior. Puesto que se carece de esa información se equipará un PAU pero no se instalará distribuidor ni tomas.

\subsubsection{Número de tomas.}
\paragraph{Número de repartidores, derivadores, según su ubicación en la red, PAU y sus características, así como las de los cables utilizados.}