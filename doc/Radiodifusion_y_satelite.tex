\subsection{Distribución de radiodifusión sonora y televisión por satélite.}

\subsection{Selección de emplazamiento y parámetros de las antenas receptoras de la señal de satélite.}

Se procederá a la instalación de dos antenas parabólicas con la orientación adecuada para captar los canales provenientes del satélite Astra e Hispasat respectivamente.

El emplazamiento previsto para ubicar las mismas es el tejado del RITU, esto queda reflejado en el plano """""""""""""""""REF"""""""""
Se ha comprobado la ausencia de obstáculos que puedan provocar obstrucción de la señal en ambos casos.


\subsection{Cálculo de los soportes para la instalación de las antenas receptoras de la señal satélite}

Para la fijación de las antenas parabólicas se construirán dos zapatas cuyas dimensiones serán definidas por el arquitecto, a las cuales se fijarán, en su día, mediante pernos de acero de 16 mm de diámetro embutidos en el hormigón que las conforma, los pedestales de las antenas.
El conjunto formado por las zapatas y los pernos de anclaje tendrá unas dimensiones y composición, a definir por el arquitecto, capaces de soportar los esfuerzos indicados en el apartado 3.1.A.a del Pliego de Condiciones.


\subsection{Mezcla de las señales de radiodifusión sonora y televisión por satélite con las terrestres.}

La señal terrestre (radiodifusión sonora y televisión) se distribuye mediante un repartidor para cada uno de los dos cables: "A" y "H". Cada una de las señales digitales de satélite correspondientes a los cables A y H se mezcla con las señales terrestres utilizando un mezclador y configurando así la señal completa para cada uno de los cables.

\subsection{Descripción de los elementos componentes de la instalación.}
Los componentes como sistemas captadores, amplificadores, cableado, etc... se encuentran detallados en el Pliego de Condiciones.
