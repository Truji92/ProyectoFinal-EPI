\subsection{Objeto del Proyecto Técnico.}
Dar cumplimiento al Real Decreto-ley 1/1.998 de 27 de Febrero sobre infraestructuras comunes en los
edificios para el acceso a los servicios de telecomunicaciones y establecer los condicionantes
técnicos que debe cumplir la instalación de ICT, de acuerdo con el Real Decreto 346/2011, de 11 de marzo, relativo al Reglamento regulador de las infraestructuras comunes de edificios y a la Roden ITC/1644/2011, de 10 de junio, del Ministerio de Industria Turismo y Comercio, que desarrolla el citado Reglamento.
Así mismo se dará cumplimiento a la LEY 10/2005, de 14 de junio (BOE 15/06/2005), de medidas urgentes para el impulso de la Televisión Digital Terrestre, de liberalización de la televisión por cable y de fomento del pluralismo.

La infraestructura común de telecomunicaciones consta de los elementos necesarios para satisfacer inicialmente las siguientes funciones:

\begin{itemize}
	\item La captación y adaptación de las señales digitales, terrestres, de radiodifusión sonora y televisión y su distribución hasta puntos de conexión situados en las distintas viviendas o locales de las edificaciones, y la distribución de las señales, por satétlite, de radiodifusión sonora y televisión hasta los citados puntos de conexión.
	\item Proporcionar el acceso a los servicios de telefonía disponible al público (STDP) y a los servicios de telecomunicaciones de banda ancha prestados a través de redes pñublicas de comunicaciones electrónicas por operadores habilitados para el establecimiento y explotación de las mismas, mediante la infraestructura necesaria que permita la conexión de las distintas viviendas o locales a las redes de los operadores habilitados.
\end{itemize}

La iCt está sustentada por la infraestructura de canalizaciones dimensionada según el Anexo III del Real Decreto 346/2011, que garantiza la posibilidad de incorporación de nuevos servicios que puedan surgir en un próximo futuro.

Se ha establecido un plan de frecuencias para la distribución de las señales de televisión y radiodifusión terrestre de las entidades con título habilitante que, sin manipulación ni conversión de frecuencias, permita la distribución de señales no contempladas en la instalación inicial por los canales previstos, de forma que no se afecten los servicios existentes y se respeten los canales destinados a otros servicios que puedan incorporarse en un futuro. La desaparición de la TV analógica y la incorporación de la TV digital terrestre conlleva el uso de las frecuencias 195.0 MHz a 223.0 MHz (C8 a C11, BIII) y 470 MHz a 862 MHz (C21 a C69, BIV y BV), que se destinarán con carácter prioritario, para la distribución de señales de radiodifusión sonora digital y televisión digital terrestre.

